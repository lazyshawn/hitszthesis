%! TEX root = ../main.tex

\conclusion
本课题的研究内容主要是从运动过程中系统的物理学模型、对应的控制方法、
实验方案、仿真实验设计和具体实验操作等方面展开,
研究并验证单自由度机械手利用重力实现在手操作的实现方法。
本文主要研究的工作总结如下:

\begin{listNormal}
  \item 分析了单自由度机器人手在手操作过程中物体的运动规律。
    对摩擦模型进行了假设, 用滑动摩擦力和粘滞摩擦力来近似表示物体
    在滑动临界点和正常运动时的摩擦力。根据牛顿定律得到物体运动过程中的动力学模型,
    分析夹爪施加给物体的正压力和物体运动状态之间的关系。
  \item 建立夹爪和被控物体系统的控制方程。根据物体运动过程中的动力学方程
    分析了在手操作的控制难点在于摩擦力的估计和考虑被控物体质量的可变性。
    参照模型参考自适应控制理论推到了系统的自适应控制方程,
    以夹爪施加的正压力为系统输入控制工具的运动状态,
    使得系统在摩擦系数未知的情况下可以对不同质量的工具进行在手操作。
  \item 分析系统控制方程, 设计控制方案。
    本研究设计并改良PD控制算法来控制夹爪施加给物体的正压力,
    保证系统控制输入量与控制方程一致。
    采用伺服控制来实现高频率运动, 保证控制操作精度。
    通过C++多线程编程协调伺服控制、力控、数据采集与处理等进程的时间安排。
  \item 设计仿真和实验方案, 验证自适应控制算法的准确性。
    调试和校准了实验设备, 并进行夹爪的在手操作实验, 实验结果不符合预期,
    原因是自适应控制参数设置不合理, 且出现了积分饱和现象。
    为此本项目使用Simulink仿真软件验证控制方程的正确性, 在此基础上搭建系统的仿真模型,
    验证了控制方案的合理性。
    同时仿真结果表明使用自适应控制后,在摩擦系数和物体质量未知的情况下,
    该控制系统仍然能正常运行,依照指定的轨迹运动到期望位置。
\end{listNormal}

针对单自由度机器人手利用重力对工具进行手上操作的实现方法这一问题,
本文先从理论角度分析了工具运动过程中的物理现象,对摩擦模型进行了假设,
建立起动力学模型。
为了建立合适的控制模型,我们选择使用模型参考自适应控制系统来建立控制模型,
并阐述了其基本原理和作用,推导了系统的控制方程和调参律。
从实验的角度分析该问题,我们设计了可实践的伺服控制算法和力控算法,
并通过实验验证该算法的合理性。
最后进行了仿真实验,实验结果验证了单自由度机器人手用于在手操作的可行性;
同时表明使用模型参考自适应控制后,在部分物理参数未知的情况下,
该控制系统仍然能正常运行,依照指定的轨迹运动到期望位置。

本文的研究结果意味着,通过使用合适的控制策略,
结构简单、抓取方式单一的平行指夹具也可以实现复杂的在手操作动作,
在实际生产线上可以有更广阔的应用范围。

鉴于时间和个人能力有限,本项目目前的实验结果尚不理想,这可能影响到本文的严谨性。
后续的工作将会继续展开,之后的主要内容是完成一系列严谨的实验,
进一步验证平行指夹具进行手上操作方面的可能性,并实现更复杂的操作。

