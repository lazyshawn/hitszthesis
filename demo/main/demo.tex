%! TEX root = ../main.tex

\chapter{常用环境的使用}
\section{公式}
牛顿第二定律描述了物体的动量和所受外力之间的关系,
动量为$p$的质点,在外力$F$的作用下,
其动量随时间的变化率同该质点所受的外力成正比,并与外力的方向相同。
牛顿第二定律的数学表述如式(\ref{equ:Newton_2})所示。
\begin{equation}
  F_e=\frac{d}{dt}mv;
  \label{equ:Newton_2}
\end{equation}

\begin{note}
  \para{$F_e$}{物体所受合外力(N);\hfill{}}
  \para{$m$}{物体质量(kg);}
  \para{$v$}{物体运动速度($\rm{m/s^2}$)。}
\end{note}

通常,物体质量不随时间变化,因此牛顿第二定律也可表述为$F_e=ma$。


\section{插图}
图\ref{fig:HK}是一个包含四个子图的浮动体。
图\ref{subfig1-1}$\sim$ \ref{subfig1-3}是三个插入题注的子图,
而第四个子图不插入题注。
% 双栏图片宽度一般60~70mm; 单栏图片宽度一般120~150mm
\begin{figure}[!ht]
\centering
  \subfloat[子图1\label{subfig1-1}]{
    \includegraphics[scale=0.07]{pics/pic.jpg}
    % \setcounter{subfigure}{0}  % 子图序号计数器
  }
  \hspace{30pt}
  \subfloat[子图2\label{subfig1-2}]{
    \includegraphics[scale=0.07]{pics/pic.jpg}
  } \\
  \subfloat[子图3\label{subfig1-3}]{
    \includegraphics[scale=0.07]{pics/pic.jpg}
  }
  \hspace{30pt}
  \subfloat{
    \includegraphics[scale=0.07]{pics/pic.jpg}
}
\caption{Hollow Knight\label{fig:HK}}
\vspace{-0.3cm}
\end{figure}


\section{表格}
三线表的绘制。很多刊物都明确规定稿件中要使用三线表。三线表在必要时也可添加若干条水平辅助分隔中线。
\begin{table}[!h]
\centering
\caption{序号计数器及其用途\label{tab:2-1}}
\begin{tabular}{@{}ll@{}}
\toprule[1pt]
计数器名   & 用途            \\ \midrule
chapter    & 章序号计数器    \\
section    & 节序号计数器    \\
subsection & 小节序号计数器  \\
\bottomrule[1pt]
\end{tabular}
\end{table}

表\ref{tab:2-1}中列出了常用章节的计数器名称。

\section{列表}
排序列表
\begin{enumerate}
  \item item of the list.
  \item item of the list.
  \item item of the list.
\end{enumerate}


